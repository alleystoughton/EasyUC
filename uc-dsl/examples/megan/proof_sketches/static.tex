\def\IsDraft{} % set for draft version

\documentclass{article}[12pt]

% == Import Packages ==
\usepackage[margin=1in]{geometry} % Page dimensions
\usepackage{amsmath} % Math formulas
\usepackage{xcolor} % Colors
\usepackage{hyperref} % Hyperlinks
\usepackage{cleveref} % Clever References

% == Inline comments ==
\ifdefined\IsDraft
\newcommand{\authnote}[2]{[{\color{red}\textbf{#1:}}~{\color{blue} #2}]}
\else
\newcommand{\authnote}[2]{}
\fi

\newcommand{\megan}[1]{\authnote{Megan}{#1}}

% == Code snippets ==
% Docs: http://texdoc.net/texmf-dist/doc/latex/listings/listings.pdf
\usepackage{listings}

\lstdefinelanguage{easycrypt}{
	morekeywords={lemma, module, proc, var, return, if, else, require, import, type, op, axiom, pred, section, local}
}

% Define easycrypt environment
\lstnewenvironment{easycrypt}[1][]
{
	\noindent
	\minipage{\linewidth}
	\vspace{0.5\baselineskip}
	\lstset{ % Set inline code parameter(s)
	  basicstyle=\small,						% print whole listing small
	  keywordstyle=\color{black}\bfseries,	% bold black keywords
	  identifierstyle=, 						% nothing happens
	  commentstyle=\color{gray}, 			% white comments
	  stringstyle=\ttfamily, 					% typewriter type for strings
	  showstringspaces=false,					% no special string spaces
	  tabsize=2,								% tabs typeset as 2 spaces
	  frame=single,							% lines above and below code snippets
	  breaklines=true,						% break lines when there's overflow
	  language=easycrypt,
	  morecomment=[n]{(*}{*)},				% define comment delimiters
	  literate={~} {$\sim$}{1},				% prettier tilde
	}
}{\endminipage}

% == Proof ==
% Entities
\newcommand{\Sim}{{\mathsf{Sim}}} % Simulator
\newcommand{\Adversary}{{\mathsf{Adv}}} % Adversary
\newcommand{\Environment}{{\mathcal{Z}}} % Environment
\newcommand{\CFPTPAdversary}{{\mathcal{D}}}
% Hybrids
\newcommand{\Ideal}{{\mathsf{Ideal}}}
\newcommand{\Hyb}{{\mathsf{Hyb}}}
\newcommand{\Real}{{\mathsf{Real}}}


% == Messages ==
\newcommand{\OpenMsg}{\mathsf{Open}}
\newcommand{\CommitMsg}{\mathsf{Commit}}

% PKE
\newcommand{\PKE}{\mathsf{PKE}}
\newcommand{\Gen}{\mathsf{Gen}}
\newcommand{\Enc}{\mathsf{Enc}}
\newcommand{\Dec}{\mathsf{Dec}}

% CFPTP
\newcommand{\CFPTP}{\mathsf{CFPTP}}
\newcommand{\Forw}{\mathsf{Forw}}
\newcommand{\Back}{\mathsf{Back}}
\newcommand{\ForwKey}{fk}
\newcommand{\BackKey}{bk}
\newcommand{\Domain}{D}

% Document title
\title{Proof sketch for \cite{CanettiF01}'s static protocol}
\author{Megan Chen}
\date{\today}

\begin{document}
\maketitle
\tableofcontents

\section{Preliminaries}

\section{Proof Strategy}
The proof proceeed via a sequence of games.

\subsection{Sequence of Hybrids}
Define $\Hyb$ to work exactly as the $\Ideal$ game, except that if the committer is \underline{not} corrupted, the simulator $\Sim_{\Hyb}$ learns the real committed bit $b$ and sends the actual commit string $(y, c_0, c_1)$.
\begin{enumerate}
	\item\label{hyb:1} $\Real \approx \Hyb$ - reduction to claw-free pair of trapdoor permutations (CFPTP).
	\item\label{hyb:2} $\Hyb \approx \Ideal$ - reduction to IND-CPA security of PKE
\end{enumerate}

The overall goal is the show that

\begin{easycrypt}
lemma REAL_IDEAL &m :
`|Pr[REAL().main() @ &m : res] - Pr[Ideal().main() @ &m : res]|
<= Pr[CFP_Game(CFAdv).main() @ &m : res]
   + `|Pr[INDCPA_0(Adv).main() @ &m : res] - Pr[INDCPA_1(Adv).main() @ &m : res]|.
\end{easycrypt}

\section{Hybrid \ref{hyb:1}}

\subsection{Overview}
We use up to bad reasoning to show an upper bound on the environment $\Environment$'s ability to distinguish between $\Real$ and $\Hyb$. In particular, the only difference in $\Environment$'s view between $\Real$ and $\Hyb$ is when $\Sim_\Hyb$ aborts. In detail, $\Sim_\Hyb$ aborts when it receives a $\CommitMsg$ message $(y, c_0, c_1)$ and an $\OpenMsg$ message $(b', x', r)$ such that
\begin{itemize}
	\item $x' = \CFPTP.\Back_{b'}(y)$
	\item $c_{b'} = \PKE.\Enc(x)$
	\item $b' \ne b$ where $b$ is the real committed bit.
\end{itemize}

This is a ``bad event'' because $\Sim_\Hyb$ found a claw for the pair of trapdoor permutations. Recall that in $\Hyb$, the $y$ in the $\CommitMsg$ message is $y = \CFPTP.\Forw_{b}(x)$, so the claw is $x, x'$.

Hence, to we upper-bound $\Adversary$'s distinguishing advantage via defining an adversary $\CFPTPAdversary$ that finds claws for a given CFPTP.

Given a CFPTP forward key $\ForwKey$, $\Adversary_{\Hyb}$ with input $y$ (from the permutation domain $\Domain$) works as follows:
\begin{itemize}
	\item Act as the simulator $\Sim_\Hyb$.
	\item If $\Sim_\Hyb$ aborts, output $x, x'$.
\end{itemize}
This succeeds with the same probability $\Sim_\Hyb$ aborts.

\subsection{Easycrypt Proof}
Since $\Environment$'s view is identical between $\Real$ and $\Hyb$ except when $\Sim_\Hyb$ aborts, most of the easycrypt proof should be a series of rewritings and transformations that do not give $\Environment$ any distinguishing advantage. Then, we can upper bound $\Environment$'s distinguishing advantage by the probability of the ``bad event'', i.e. when $\Sim_\Hyb$ finds a claw for the pair of trapdoor permutations. We can express this via the following lemma:

\begin{easycrypt}
lemma REAL_HYB &m :
`|Pr[REAL.main() @ &m : res] - Pr[HYB.main() @ &m : res]|
<= Pr[CFP_Game(CFAdv).main() @ &m : res].
\end{easycrypt}

We also need the following helper lemmas:

\begin{easycrypt}
lemma REAL_HYB_main &m :
equiv[REAL.main ~ HYB.main : true => (* If Sim_Hyb didn't abort, then programs are equivalent. *)].

lemma HYB_main_abort_ub &m :
Pr[HYB.main() @ &m : (* Sim_Hyb aborts *)] <= Pr[CFP_Game(CFAdv).main() @ &m : res].
\end{easycrypt}

To prove this, we define a claw-finding adversary. First, in ${\sf Cfptp.ec}$, we defined the claw-free pair game is defined as

\begin{easycrypt}
module CFP_Game(Cf: ClawFinder) = {
	proc main(): bool = {
		var fk: fkey; var bk: bkey;
		var x0, x1 : D;

		(fk, bk) <$ keygen;             (* Generates keys for CFPTP *)
		(x0, x1) <@ Cf.find_claw(fk);   (* Find any claw for the CFPTP *)
		return (forw fk x0 false = forw fk x1 true); (* Cf succeeds when this happens *)
	}
}.
\end{easycrypt}

Now, define the claw-finding module type:

\begin{easycrypt}
module type ClawFinder = {
  proc find_claw(fk: fkey) : (D * D)
}.
\end{easycrypt}

Next, define a claw-finding adversary which simulates $\Sim_\Hyb$'s execution and outputs a claw if $\Sim_\Hyb$ aborts.

\begin{easycrypt}
module CFAdv : ClawFinder ( (* has access to Sim_Hyb *) ) = {
	var Sim.x, Sim.x' : D;
	proc find_claw(fk : fkey) : D * D = {

		(* Simulate running Sim_Hyb interacting with an adversary when fk is in the CRS. *)

		if ( (* Sim_Hyb aborts *) ) {
			(* Finding a claw means: forw fk x false = forw fk x true = y *)
			return (Sim.x, Sim.x'); (* Get and return the claw from Sim's memory. *)
		}
		else {
			return (Sim.y, Sim.y); (* Failure to find claw *)
		}
	}
}.
\end{easycrypt}

{\small{
\bibliographystyle{alpha}
\bibliography{refs}
}}

\end{document}
