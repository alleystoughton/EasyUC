\def\IsDraft{} % set for draft version

\documentclass{article}[12pt]

% == Import Packages ==
\usepackage[margin=1in]{geometry} % Page dimensions
\usepackage{amsmath} % Math formulas
\usepackage{xcolor} % Colors
\usepackage{hyperref} % Hyperlinks
\usepackage{cleveref} % Clever References

% == Inline comments ==
\ifdefined\IsDraft
\newcommand{\authnote}[2]{[{\color{red}\textbf{#1:}}~{\color{blue} #2}]}
\else
\newcommand{\authnote}[2]{}
\fi

\newcommand{\alley}[1]{\authnote{Alley}{#1}}
\newcommand{\megan}[1]{\authnote{Megan}{#1}}
\newcommand{\mayank}[1]{\authnote{Mayank}{#1}}
\newcommand{\ran}[1]{\authnote{Ran}{#1}}

% == Code snippets ==
% Docs: http://texdoc.net/texmf-dist/doc/latex/listings/listings.pdf
\usepackage{listings}

\lstdefinelanguage{easycrypt}{
	morekeywords={lemma, module, proc, var, return, if, else, require, import, type, op, axiom, pred, section, local}
}

% Define easycrypt environment
\lstnewenvironment{easycrypt}[1][]
{
	\noindent
	\minipage{\linewidth}
	\vspace{0.5\baselineskip}
	\lstset{ % Set inline code parameter(s)
	  basicstyle=\small,						% print whole listing small
	  keywordstyle=\color{black}\bfseries,	% bold black keywords
	  identifierstyle=, 						% nothing happens
	  commentstyle=\color{gray}, 			% white comments
	  stringstyle=\ttfamily, 					% typewriter type for strings
	  showstringspaces=false,					% no special string spaces
	  tabsize=2,								% tabs typeset as 2 spaces
	  frame=single,							% lines above and below code snippets
	  breaklines=true,						% break lines when there's overflow
	  language=easycrypt,
	  morecomment=[n]{(*}{*)},				% define comment delimiters
	  literate={~} {$\sim$}{1},				% prettier tilde
	}
}{\endminipage}

\newcommand{\code}[1]{\texttt{#1}}%inline code

% == Messages ==
\newcommand{\OpenMsg}{\mathsf{Open}}
\newcommand{\CommitMsg}{\mathsf{Commit}}

% PKE
\newcommand{\PKE}{\mathsf{PKE}}
\newcommand{\Gen}{\mathsf{Gen}}
\newcommand{\Enc}{\mathsf{Enc}}
\newcommand{\Dec}{\mathsf{Dec}}
\newcommand{\Indcpa}{\mathsf{INDCPA}}
\newcommand{\IndcpaGM}{\mathcal{C}} % the game master (i.e. challenger) for the indcpa game.

% CFPTP
\newcommand{\CFPTP}{\mathsf{CFPTP}}
\newcommand{\Forw}{\mathsf{Forw}}
\newcommand{\Back}{\mathsf{Back}}
\newcommand{\ForwKey}{fk}
\newcommand{\BackKey}{bk}
\newcommand{\Domain}{D}


% == Proof ==
% Entities
\newcommand{\Sim}{{\mathsf{Sim}}} % Simulator
\newcommand{\Adversary}{{\mathsf{Adv}}} % Adversary
\newcommand{\Environment}{{\mathcal{Z}}} % Environment
\newcommand{\CFPTPAdversary}{{\Adversary_\CFPTP}}
\newcommand{\IndcpaAdversary}{{\Adversary_\Indcpa}}

% Hybrids
\newcommand{\Ideal}{{\mathsf{Ideal}}}
\newcommand{\Hyb}{{\mathsf{Hyb}}}
\newcommand{\Real}{{\mathsf{Real}}}


% == Document title ==
\title{Proof sketch for \cite{CanettiF01}'s static protocol}
\author{Megan Chen}
\date{\today}

\begin{document}
\maketitle
\tableofcontents

\section{Preliminaries}

\section{Proof Strategy}
The proof proceeed via a sequence of games.

\subsection{Sequence of Hybrids}
Define $\Hyb$ to work exactly as the $\Ideal$ game, except that if the committer is \underline{not} corrupted, the simulator $\Sim_{\Hyb}$ learns the real committed bit $b$ and sends the actual commit string $(y, c_0, c_1)$. \megan{expand $\Hyb$ description}
\begin{enumerate}
	\item\label{hyb:1} $\Real \approx \Hyb$ - reduction to claw-free pair of trapdoor permutations (CFPTP).
	\item\label{hyb:2} $\Hyb \approx \Ideal$ - reduction to IND-CPA security of PKE
\end{enumerate}

The overall goal is the show that

\begin{easycrypt}
lemma REAL_IDEAL &m :
`|Pr[REAL().main() @ &m : res] - Pr[Ideal().main() @ &m : res]|
<= Pr[CFP_Game(CFAdv).main() @ &m : res]
   + `|Pr[INDCPA_0(INDCPAAdv).main() @ &m : res] - Pr[INDCPA_1(INDCPAAdv).main() @ &m : res]|.
\end{easycrypt}

\section{Hybrid \ref{hyb:1}: Showing $\Real \approx \Hyb$}

\subsection{Overview}
We use up to bad reasoning to show an upper bound on the environment $\Environment$'s ability to distinguish between $\Real$ and $\Hyb$. In particular, the only difference in $\Environment$'s view between $\Real$ and $\Hyb$ is when $\Sim_\Hyb$ aborts. In detail, $\Sim_\Hyb$ aborts when it receives a $\CommitMsg$ message $(y, c_0, c_1)$ and an $\OpenMsg$ message $(b', x', r)$ such that
\begin{itemize}
	\item $x' = \CFPTP.\Back_{b'}(y)$
	\item $c_{b'} = \PKE.\Enc_{pk}(x)$
	\item $b' \ne b$, where $b$ is the bit $\Sim_\Hyb$ detects from the $\CommitMsg$ message.
\end{itemize}

This is a ``bad event'' because $\Sim_\Hyb$ found a claw for the pair of trapdoor permutations. Recall that in $\Hyb$, the $y$ in the $\CommitMsg$ message is $y = \CFPTP.\Forw_{b}(x)$, so the claw is $x, x'$.

Hence, to we upper-bound $\Adversary$'s distinguishing advantage via defining an adversary $\CFPTPAdversary$ that finds claws for a given CFPTP.

Given a CFPTP forward key $\ForwKey$, $\Adversary_{\Hyb}$ with input $y$ (from the permutation domain $\Domain$) works as follows:
\begin{itemize}
	\item Act as the simulator $\Sim_\Hyb$.
	\item If $\Sim_\Hyb$ aborts, output $x, x'$.
\end{itemize}
This succeeds with the same probability $\Sim_\Hyb$ aborts.

\subsection{Easycrypt Proof}
Since $\Environment$'s view is identical between $\Real$ and $\Hyb$ except when $\Sim_\Hyb$ aborts, most of the easycrypt proof should be a series of rewritings and transformations that do not give $\Environment$ any distinguishing advantage. Then, we can upper bound $\Environment$'s distinguishing advantage by the probability of the ``bad event'', i.e. when $\Sim_\Hyb$ finds a claw for the pair of trapdoor permutations. We can express this via the following lemma:

\begin{easycrypt}
lemma REAL_HYB &m :
`|Pr[REAL.main() @ &m : res] - Pr[HYB.main() @ &m : res]|
<= Pr[CFP_Game(CFAdv).main() @ &m : res].
\end{easycrypt}

\megan{\code{HYB} is not currently written anywhere. It can be expressed either using the DSL or done directly in the generated easycrypt code.}

We also need the following helper lemmas:

\begin{easycrypt}
lemma REAL_HYB_main &m :
equiv[REAL.main ~ HYB.main : true => (* If Sim_Hyb didn't abort, then programs are equivalent. *)].
\end{easycrypt}

\megan{capture $\Sim_\Hyb$ aborting with a top-level variable.}
\megan{maybe add a diagram to show that the programs stay the same, except when bad event happens}

\begin{easycrypt}
lemma HYB_main_abort_ub &m :
Pr[HYB.main() @ &m : (* Sim_Hyb aborts *)] <= Pr[CFP_Game(CFAdv).main() @ &m : res].
\end{easycrypt}

\megan{Add intermediate step for eager sampling}

To prove this, we define a claw-finding adversary. First, in ${\sf Cfptp.ec}$, we defined the claw-free pair game is defined as

\begin{easycrypt}
module CFP_Game(Cf: ClawFinder) = {
	proc main(): bool = {
		var fk: fkey; var bk: bkey;
		var x0, x1 : D;

		(fk, bk) <$ keygen;             (* Generates keys for CFPTP. $*)
		(x0, x1) <@ Cf.find_claw(fk);   (* Find any claw for the CFPTP *)
		return (forw fk x0 false = forw fk x1 true); (* Cf succeeds when this happens *)
	}
}.
\end{easycrypt}

Now, define the claw-finding module type:

\begin{easycrypt}
module type ClawFinder = {
	proc find_claw(fk: fkey) : (D * D)
}.
\end{easycrypt}

Next, define a claw-finding adversary which runs $\Sim_\Hyb$'s execution and outputs a claw if $\Sim_\Hyb$ aborts.

\begin{easycrypt}
module CFAdv : ClawFinder ( (* has access to Environment for Sim vs Hyb game *) ) = {
	var Sim.x, Sim.x' : D;
	proc find_claw(fk : fkey) : D * D = {

		(* Run environment for Sim vs Hyb game, with fk in the CRS. *)

		if ( (* Environment sees that Sim_Hyb aborts *) ) {
			(* Finding a claw means: forw fk x false = forw fk x true = y *)
			return (Sim.x, Sim.x'); (* Get and return the claw from Sim's memory. *)
		}
		else {
			return (Sim.y, Sim.y); (* Failure to find claw. Equivalent to randomly guessing two elements. *)
		}
	}
}.
\end{easycrypt}

\section{Hybrid \ref{hyb:2}: Showing $\Hyb \approx \Ideal$}

\subsection{Overview} \label{sec:indcpa_red_overview}
IND-CPA (i.e. semantic) security in a PKE scheme captures the property that the scheme's ciphertexts reveal only a negligible amount of information about their underlying plaintexts. We show that any adversary that distinguishes between $\Hyb$ and $\Ideal$ can be used to win the IND-CPA game (as described in \cref{sec:indcpa}) via a cryptographic reduction.

The outline for the reduction works as follows: Let $p$ be the probability of distinguishing ciphertexts in some IND-CPA PKE scheme. Assume (for a contradiction) that there exists an $\Environment$ distinguishing between $\Hyb$ and $\Ideal$ with probability $> p$. Next, construct an adversary $\IndcpaAdversary$ that wins the IND-CPA security game with a probability that is too high, contradicting the IND-CPA security of the PKE scheme. Thus, we conclude that $\Environment$'s distinguishing advantage is $\leq p$.

First, we describe the difference between $\Hyb$ and $\Ideal$. \megan{Add: argue that apart from $c_{1-b}$, Hyb and Ideal give the same view.}
In $\Hyb$, observe that the $\CommitMsg$ message $(y, c_0, c_1)$ is
\begin{itemize}
	\item $y = \CFPTP.\Forw_{b}(x)$
	\item $c_b = \PKE.\Enc_{pk}(x; r_b)$
	\item $c_{1-b} = \PKE.\Enc_{pk}(0; r_{1-b})$.
\end{itemize}
where $x, r_0, r_1$ are uniformly sampled.

In $\Ideal$, the $\CommitMsg$ message $(y, c_0, c_1)$ is:
\begin{itemize}
	\item $y = \CFPTP.\Forw_{b}(x_b)$
	\item $c_b = \PKE.\Enc_{pk}(x_b; r_b)$
	\item $c_{1-b} = \PKE.\Enc_{pk}(x_{1-b}; r_{1-b})$
\end{itemize}
where $x_0, x_1, r_0, r_1$ are uniformly sampled.

The difference between $\Hyb$ and $\Ideal$ is the encrypted plaintext for $c_{1-b}$. In essence, distinguishing $\Hyb$ and $\Ideal$, means distinguishing between whether $c_{1-b}$ is an encryption of $0$ or $x_{1-b}$.

Hence, given an environment $\Environment$ that distinuguishes between $\Hyb$ and $\Ideal$ with non-negligible probability $>p$, we can construct an adversary $\IndcpaAdversary$ that breaks IND-CPA security. On input the committed bit $b$, $\IndcpaAdversary$ interacts with to a challenger $\IndcpaGM$ and does the following:
\begin{enumerate}
	\item Sample $x_0, x_1, r_0, r_1$ uniformly.
	\item Honestly compute $y = \CFPTP.\Forw_b(x_b)$ and $c_b = \PKE.\Enc_{pk}(x_b; r_b)$.
	\item Send $(0, x_{1-b})$ to $\IndcpaGM$, and receive an encryption $c^*$ of either $0$ or $x_{1-b}$.
	\item Set $c_{1-b} = c^*$.
	\item \megan{Generate the rest of $\Environment$'s view (incl. $\OpenMsg$ message)}.
	\item Send $(y, c_b, c_{1-b})$ to $\Environment$, which outputs a bit indicating ``$\Hyb$'' or ``$\Ideal$''.
	\item If $\Environment$ outputs ``$\Hyb$'' output 0. If $\Environment$ outputs ``$\Ideal$'' output 1.
\end{enumerate}

Here, if $\Environment$ sees that $c_{1-b} = \PKE.\Enc_{pk}(0; r_{1-b})$, it will output $\Hyb$. When $\Environment$ sees that $c_{1-b} = \PKE.\Enc_{pk}(x_{1-b}; r_{1-b})$, it will output $\Ideal$. Hence, $\IndcpaAdversary$ outputs the correct bit with exactly the same probability that $\IndcpaAdversary$ distinguishies between $\Hyb$ and $\Real$. Then since $p$ is non-negligible, this is a contradiction.

\subsection{Easycrypt Proof}
Following the logic from \cref{sec:indcpa_red_overview}, our goal is to show that the $\Environment$'s distinguishing advantange between $\Hyb$ and $\Ideal$ is:

\begin{easycrypt}
lemma HYB_IDEAL &m :
`|Pr[HYB.main() @ &m : res] - Pr[IDEAL.main() @ &m : res]|
<= `|Pr[INDCPA_0(INDCPAAdv).main() @ &m : res] - Pr[INDCPA_1(INDCPAAdv).main() @ &m : res]|.
\end{easycrypt}

\subsubsection{IND-CPA Game}\label{sec:indcpa}
The IND-CPA game is formalized in ${\sf Pke.ec}$ via defining two modules \code{INDCPA\_0}, \code{INDCPA\_1} that resspectively encrypt two adversarially chosen plaintexts $x_0, x_1$:

\begin{easycrypt}
module INDCPA_0(INDCPAAdv : ADV_INDCPA) = { (* Always encrypts x0 *)
	proc main() : bool = {
		var pk : pkey; var sk : skey;
		var r : rand;
		var x0, x1 : plaintext;
		var b : bool;
		var c : ciphertext;
		(pk, sk) <$ dkeygen;         		(* Generate keys for PKE *)
		r <$ drand;                  		(* Select randomness used in PKE *)
		(x0, x1) <@ INDCPAAdv.choose(pk);	(* Adv chooses plaintexts x0, x1 *)
		c <- enc pk x0 r;            		(* Encrypt x0 *)
		b <@ INDCPAAdv.main(pk, c);			(* Adv guesses which ciphertext was encrypted *)
		return b;
	}
}.

module INDCPA_1(INDCPAAdv : ADV_INDCPA) = { (* Always encrypts x1*)
	proc main() : bool = {
		var pk : pkey; var sk : skey;
		var r : rand;
		var x0, x1 : plaintext;
		var b : bool;
		var c : ciphertext;
		(pk, sk) <$ dkeygen;         		(* Generate keys for PKE *)
		r <$ drand;                  		(* Select randomness used in PKE *)
		(x0, x1) <@ INDCPAAdv.choose(pk);	(* Adv chooses plaintexts x0, x1 *)
		c <- enc pk x1 r;            		(* Encrypt x1 *)
		b <@ INDCPAAdv.main(pk, c);			(* Adv guesses which ciphertext was encrypted *)
		return b;
	}
}.
\end{easycrypt}

Then, an IND-CPA adversary wins if it correctly guesses which module (\code{INDCPA\_0}, \code{INDCPA\_1}) it is in. We formalize this idea as the adversary's ``advantage'', which is captured as:

\begin{easycrypt}
`|Pr[INDCPA_0(INDCPAAdv).main() @ &m : res] - Pr[INDCPA_1(INDCPAAdv).main() @ &m : res]|
\end{easycrypt}

Next, we build a module for the INDCPA adversary \code{INDCPAAdv}:

\begin{easycrypt}
module INDCPAAdv : ADV_INDCPA ( (* has access to the Environment for Hyb vs Ideal game *) ) = {
	var c : plaintext;
	proc choose(pk : pkey) : plaintext * plaintext = {
		x <$ dplaintext;	(* Sample a random plaintext $*)
		return (zero, x)	(* Return two plaintexts: 0 and the one sampled above. *)
	}

	proc main(pk : pkey, c : ciphertext) : bool = {
		(* Query the Hyb / Ideal distinguishing Environment with c. Receive Environment's output guessing whether it's in Hyb or Ideal *)

		if ( (* Environment outputs ``Hyb'' *) ) {
			return 0;	(* Guess that c is an encryption of 0 *)
		}
		else ( (* Environment outputs ``Ideal'' *) ) {
			return 1;	(* Guess that c is an encryption of some plaintext x *)
		}

	}
}.
\end{easycrypt}

{\small{
\bibliographystyle{alpha}
\bibliography{refs}
}}

\end{document}
